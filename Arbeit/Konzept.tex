\chapter{System Konzeption} 
\label{ch:konzeption} 
Nachdem in den ersten Kapiteln dieser Arbeit die grundlegenden fachspezifischen Anforderungen der Hochschulsporteinrichtungen an ein Softwaresystem sowie einige wichtige Konzepte der Systementwicklung vorgestellt wurden, soll im vierten Teil ein Konzept erstellt werden, wie eine Umsetzung unter Betrachtung der Anforderungen aussehen kann. Des Weitern sollen in dieses Konzept Anforderungen aus Sicht  des Software Erstellers mit berücksichtigt werden, sodass ein einstehendes Konzept theoretisch auch in der Praxis umgesetzt werden kann. \\
Der Aufbau der Konzeption erfolgt dabei in Anlehnung an \citet*[]{Balzert.1996} und beinhaltet die Phasen:
\begin{enumerate}
\item Planungsphase
\item Definitionsphase
\item Entwurfsphase
\end{enumerate}

\section{Planungsphase}
- Lastenheft \\
  - Nutzeranforderungen\\
  - Systemanforderungen\\
	  - Multi-Tenantcy\\
	  - PaaS, SaaS\\
 - Glossar\\
\\
- WAS und nicht WIE\\
- Priorisierung?\\
\section{Definitionsphase}
- Pflichtenheft\\
- \\
\section{Entwurfsphase}
- WIE\\
- Entwerfen einer Software-Architektur\\
	- Zerlegung des definierten Systems in Systemkomponenten\\
	- Strukturierung des Systems durch geeignete Anordnung der Systemkomponenten\\
	- Beschreibung der Beziehungen zwischen den Systemkomponenten\\
- Festlegung der Schnittstellen, über die die Systemkomponenten miteinander kommunizieren.\\
\\
Microservice Diagram\\
https://dzone.com/articles/building-microservices-inter-process-communication-1\\
https://insidethecpu.com/2015/07/17/microservices-in-c-part-1-building-and-testing\\