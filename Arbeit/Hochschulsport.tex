\section{Analyse der Fachdomäne Hochschulsport} 
\label{ch:hochschulsport} 
Die Aufgaben und Anforderungen, die an deutsche Hochschulen gestellt werden sind vielfältig und erstrecken sich in viele verschiedene Bereiche. Neben den Kernaufgaben universitäre Lehre und wissenschaftliche Forschung gehört es auch zu den Aufgaben das studentische Leben in sozialen und gesellschaftlichen Belangen zu unterstützen und fördern. Neben Einrichtungen wie dem Studierendenwerk oder Allgemeinen Studierendenausschuss (AStA) ist auch der Hochschulsport eine Einrichtung die unterschiedlichste Aufgaben im Umfeld der Fachhochschulen und Universitäten übernimmt. Dieser Teil soll ein Verständnis für die Domäne Hochschulsport und deren Entwicklung, Organisationsformen, Aufgaben sowie Probleme und Herausforderungen vermitteln.

\subsection{Historische Entwicklung des Hochschulsports}
Das Sporttreiben an deutschen Hochschulen begann zwar nicht erst nach dem zweiten Weltkrieg, die historische Betrachtung soll sich hier jedoch auf die Zeit zwischen 1948 und 2009 beschränken, um den Rahmen der Arbeit nicht zu sprengen. Da die Organisation von Bildung in die Zuständigkeit der Bundesländer fällt, hat die Entwicklung durchaus unterschiedliche Wege genommen. Ich konzentriere mich in diesem Fall auf den geschichtlichen Rückblick mit dem Schwerpunkt NRW, da hier die umfangreichsten Dokumentationen verfügbar waren. \\


\subsection{Typische Aufgaben}
\subsection{Organisationsformen}
\subsection{Befragung ausgewählter Standorte}
\subsection{Wichtige Geschäftsprozesse}
\subsection{Probleme und Herausforderungen}