\chapter{Analyse der Fachdomäne Hochschulsport} 
\label{ch:hochschulsport} 
Die Aufgaben und Anforderungen, die an deutsche Hochschulen gestellt werden sind vielfältig und erstrecken sich in viele verschiedene Bereiche. Neben den Kernaufgaben der universitären Lehre und wissenschaftlichen Forschung gehört es auch zu den Aufgaben, das studentische Leben in sozialen und gesellschaftlichen Belangen zu unterstützen und zu fördern. Neben Einrichtungen wie dem Studierendenwerk oder dem Allgemeinen Studierendenausschuss (AStA) ist auch der Hochschulsport eine Einrichtung, die unterschiedlichste Aufgaben im Umfeld der Fachhochschulen und Universitäten übernimmt. Im Vergleich zu amerikanischen Universitäten erscheint der Hochschulsport in Deutschland oft als rein schmückendes Element. Allein der Mitteleinsatz für die Erreichung eines hohen Niveaus des Hochschulsportprogrammes in Amerika übersteigt die Ausgaben um ein Vielfaches. In deutschen Hochschulsporteinrichtungen lagen die Aufwendungen für Personal-und Sachmittel im Jahre 2004 für Universitäten bei etwa 300.000€ und bei Fachhochschulen bei rund 30.000€ \cite[vgl. ][S.9]{Hachmeister.2004}. 
Der erste Teil dieser Diplomarbeit soll ein Verständnis für die Domäne Hochschulsport und deren Entwicklung, Organisationsformen, Aufgaben sowie Probleme und Herausforderungen vermitteln.

\section{Historische Entwicklung des Hochschulsports}
Das Sporttreiben an deutschen Hochschulen begann zwar nicht erst nach dem zweiten Weltkrieg, die historische Betrachtung soll sich hier jedoch auf die Zeit zwischen 1948 und 2009 beschränken, um den Rahmen der Arbeit nicht zu sprengen. Da die Organisation von Bildung in die Zuständigkeit der Bundesländer fällt, hat die Entwicklung durchaus unterschiedliche Wege genommen. Ich konzentriere mich in diesem Fall auf den geschichtlichen Rückblick mit dem Schwerpunkt NRW, da hier die umfangreichsten Dokumentationen verfügbar waren. 
Der 1999 von von 29 Bildungsministern initiierte Bologna-Prozess zur Schaffung eines einheitlichen Hochschulraums \cite*[vgl.][]{Zandonella.2005}
 
\subsection{Hochschulsport heute}
Durch die gesetzliche Verankerung im Hochschlrahmengesetz und die Festschreibung in den einzelnen Landeshochschulgesetzen hat sich eine vorerst gesicherte Grundlage zur Förderung des Sports an Hochschulen manifestiert. Die einzelnen Hochschulsporteinrichtungen (HSP) besitzen eine gefestigte Position im Hochschulkontext und müssen nicht mehr um ihre grundlegende Daseinsberechtigung kämpfen. Trotz gesicherter Grundlagen haben etwa 40  \% der Hochschulsporteinrichtungen mit Mittelkürzungen zu kämpfen \cite[][S.9]{Hachmeister.2004}.
Das Grundanliegen des HSP ist und bleibt jedoch die Förderung des Sporttreibens der Studierenden, die sich in speziellen Lebensphasen befinden und durch besondere Umstände geprägt sind. Das diese nicht generalisiert werden können, lässt sich leicht nachvollziehen, da unterschiedliche Studiengänge, Wohnumstände und Studiumsformen die Lebensphase in sehr heterogener Weise beeinflussen können. Der Einfluss auf den Hochschulsport und im speziellen auf die Akzeptanz und Teilnahme lässt sich an folgenden Kriterien festmachen, die speziell bei der Hochschulsportorganisation explizit berücksichtigt werden müssen:

\begin{enumerate}
	\item Sportinteresse des Studierenden
	\item Ort der Sporteinrichtungen
	\item Soziale Kontakte
	\item Belastungen und Entbehrungen
\end{enumerate}

Somit ergibt sich für den Hochschulsport heute die besondere Herausforderung den Bedürfnissen der Studierenden gerecht zu werden und ihnen die Gelegenheit zu geben, gemäß ihrer Interessen und Vorstellungen Sport treiben zu können. \\
Der bereits erwähnte Bologna-Prozess führte allerdings zu weiteren Herausforderungen im Verhältnis Hochschulsport-Studierende \cite[vgl. ][S.7]{Berthold.2007}. Die Verkürzung der Studienzeit führte zu einem strenger vorgegebenen Studienplan und einer deutlichen Verknappung der persönlichen Gestaltungs- und Freizeitmöglichkeiten. Diese Faktoren beeinflussen die Möglichkeiten der Studierenden an der Teilnahme im Hochschulsport erheblich, sodass sich des Hochschulsportprogramm diesen speziellen Anforderungen noch stärker durch mehr Flexibilität und neue Angebotsformen stellen muss.
\\
Neben dem Einfluss des Bologna-Prozesses auf die Studierenden zeigt die einher gehende Restrukturierung der Hochschulen auch direkten Einfluss auf die Hochschulsporteinrichtungen. Durch die Messung der Hochschulen anhand ihrer Erfolge in den Kernprodukten, schrumpfen die Möglichkeiten für zusätzliche Angebote, was sich auch in der Bereitstellung der Finanz- und Sachmittel widerspiegelt. \citeauthor{Berthold.2007} sehen dabei folgende vier Bedrohungen für den Hochschulsport:
\begin{enumerate}
\item \textbf{Einsparungen:} Linear an der allgemeinen Sparbemühungen beteiligt zu sein, kann den Hochschulsport bei bisher generell eher magerer Ausstattung vollends marginalisieren.
\item \textbf{Leistungsorientierung:} je mehr die Hochschulen von den Länder über eine leistungsorientierte Mittelverteilung finanziert werden, desto eher könnten die Hochschulen einen Anreiz empfinden, sich auf die in Kennzahlen ausgedrückten Kernleistungen zu konzentrieren.
\item \textbf{Studienzeitverkürzung:} Wenn im Rahmen der Umstellung auf gestufte Studiengänge tatsächlich eine striktere Einhaltung der Regelstudienzeit erreicht werden sollte, dann werden die Studierenden intensiver studieren. Das dürft dazu führen, dass sie weniger Zeit für Mitwirkung am Sport zur Verfügung haben und das sie sich weniger ehrenamtlich in der Betreuung engagieren.
\item \textbf{Studiengebühren:} Ist bereits das Studium kostenpflichtig, so werden sich die Studierenden eher fragen, ob sie für Sportangebote zusätzlich zahlen wollen. Sollten sie in den Konflikt zwischen intensiver Belastung durch das Studium und seiner Finanzierung durch Nebentätigkeiten geraten, so dürfte die Betätigung im Sport das sein, was geopfert würde.
\end{enumerate}
\cite[siehe ][S.16]{Berthold.2007}
\\
Nach der bundesweiten Abschaffung der Studiengebühren ist diese Bedrohung zur Zeit nicht mehr so prekär, wie einst angenommen.

Neben den Studierenden gehören aber zunehmend auch andere Personenkreise zur Zielgruppe des Hochschulsports. Besonders die Bediensteten der Hochschule werden zunehmend in den Programmen mit speziellen Angeboten angesprochen. Speziell im Bereich Gesundheitssport haben sich verschiedene Programme an den Hochschulen oder auch durch übergeordnete Gremien, wie den Allgemeinen Deutschen Hochschulsportverband (adh), etabliert. Das Angebot ist dabei sehr unterschiedlich und stark abhängig von den verfügbaren Ressourcen, erstreckt sich von der Teilnahme am allgemeinen Programm über spezielle Bediensteten Kurse bis hin zur individuellen Betreuung am Arbeitsplatz im Büro. [\textbf{Entwicklung von Gesundheitskursen}] Die Nachfrage nach gesundheitsfördernden Maßnahmen dieser Art erfreut sich derzeit einer hohen Nachfrage und wird sowohl vom adh als auch von den Hochschulsporteinrichungen in der Mehrheit als aktuelle und zukünftige Kernaufgabe des Hochschulsports gesehen \cite[vgl.][]{Baumgarten.2009}.
Die Potenziale des Hochschulsports, der wichtige Beitrag zur sozialen und integrativen Unterstützung sowie die positiven, gesundheitsfördernden Effekte auf die Studierenden werden von \cite{Goring.2010} zusammen gefasst. 

Neben den Bediensteten lassen sich aber auch externe Teilnehmer als weitere Zielgruppe bestimmen. Die Einbindung dieser Personenkreise unterscheidet sich am stärksten in den unterschiedlichen Einrichtungen und birgt großes Potenzial und Gefahr gleichzeitig. Je nach Organisationsform stehen dem Hochschulsport ganz unterschiedliche Möglichkeiten offen aber auch das regionale Umfeld muss hier analysiert werden. Mit der Öffnung des Hochschulsports für externe Teilnehmer begibt man sich in direkte Konkurrenz mit kommerziellen Anbietern. Die Integrationsformen variieren von Kooperationen mit Vereinen, Firmen im Rahmen von Betriebssportangeboten, Rehakliniken und Gesundheitszentren bis hin zur freien Öffnung für Jedermann. Allgemein lässt sich jedoch beobachten, dass diese Gruppe vor allem dann angesprochen wird, wenn eine Auslastung durch Bedienstete und Studierende nicht möglich ist. Im Bereich Organisationsformen werde ich auf die Unterschiede noch detaillierter eingehen.  
\section{Typische Aufgaben}
Wie bereits beschrieben, bewegt sich der Hochschulsport in einem Spannungsfeld unterschiedlicher Interessen. Die Definition der Aufgaben sind im Hochschulrahmengesetz \cite{BundsministeriumderJustizundfurVerbracherschutz.1976} nur sehr allgemein beschreiben, um den Hochschulschulen einen gewissen Freiraum in der Prioritätensetzung und Umsetzung zu belassen. Auf Basis des CHE Berichtes \textit{Hochschulsport 2004} \cite*[vgl. ][S.26f]{Hachmeister.2004} lassen sich folgende Aufgaben des Hochschulsports definieren:

\paragraph*{Dauerhafte Motivation und Möglichkeit zum Sporttreiben} 
Um eine Verbesserung der psychischen und physischen Belastbarkeit von Studierenden und Hochschulangehörigen zu erreichen, ist es wichtig diese dauerhaft zum Sport treiben zu motivieren. Dafür ist es zum einen wichtig ihnen den Spaß am Sport zu vermitteln und ihnen die Möglichkeiten zu geben, ihren Sport nach ihren Vorstellungen zu betreiben.
\paragraph*{Gesundheitsförderung und entwickeln einer umfassenden Mitverantwortung für eine gesunde Lebensführung}
Gesundheit hat viele Facetten und Sport ist eine darunter. Daher ist es wichtig zu vermitteln welche Auswirkungen sportliche Betätigung auf unser Lebensweise hat und wie wir den Sport gesund gestalten können. Darüber hinaus kann es in diesem Zusammenhang auch Aufgabe des Hochschulsports sein ein Bewusstsein für weiter Kernpunkte einer gesunden Lebensweise zu schaffen wie z.B. Ernährung, rauchen, Alkohol etc.
\paragraph*{Verbesserung der Kommunikation unter Hochschulangehörigen}
Hochschulen sind in den meiste Fällen stark hierarchisch strukturiert. Dies erschwert häufig eine entspannte Kommunikation zwischen Bediensteten unterschiedlicher Hierarchieebenen. Im Umfeld Sport werden diese Barrieren aufgebrochen und tragen zu einer besseren, entspannten Kommunikation über Ebenen hinweg bei.
\paragraph*{Bereicherung des studentischen Lebens an der Hochschule und Integration über den Sport}
Neben der Überwindung von Hochschulhierarchien sieht der Hochschulsport eine weiter Aufgabe in der Integration neuer Studierender in die studentische Gemeinschaft über gemeinsame sportliche Beschäftigungen.
\paragraph*{Profilbildung Hochschule}

\paragraph*{Erprobung neuer Trends}
Trendsportarten kommen und gehen. Für Vereine und Verbände ist es häufig schwierig neue Bewegungen direkt auf zu greifen und in eine rentables Programm zu überführen. Der Hochschulsport ist hier mit seiner Flexibilität in der Lage Trends frühzeitig auszuprobieren und an einem experimentierfreudigem Publikum zu erproben.
\paragraph*{Ausbildung von Übungsleitern}
Die Ausbildung eigener Übungsleiter ist eine häufig unterschätzte Aufgabe der Hochschulsporteinrichtungen. Da die Zufriedenheit der Teilnehmer mit der Qualifizierung der Übungsleiter korreliert, ist es wichtig für entsprechende Aus- und Weiterbildungen zu sorgen. Zudem erlernen Übungsleiter in ihrem Engagement eine Reihe von Soft-Skills, die ihnen für die Zukunft persönliche Fähigkeiten bereitstellen.

\section{Organisationsformen}
Wie bereits bemerkt lassen sich unterschiedliche Organisationsformen der Hochschuleinrichtungen finden. \citet*[S. 15]{Radde.1996} gliedert diese in drei Formen:
\paragraph*{Die zentralen Einrichtungen}
Sie ist die am häufigsten anzutreffende Organisationsform seit der Festschreibung des Hochschulsportes im Hochschulrahmengesetz. Unabhängig von einem Fachbereich und direkt dem Senat unterstellt hat sie die Aufgabe den Hochschulsport eigenständig zu organisieren und zu verwalten.

\paragraph*{Die in das Aufgabenfeld der Institute für Sportwissenschaft integrierte Organisation des Hochschulsports}
Diese enge Verbindung mit der Sportlehrer/innenausbildung kann sich auf das Hochschulsportangebot äußerst positiv auswirken, bringt aber auch durch notwendige „Prioritätensetzungen“ die Gefahr der Benachteiligung mit sich. Nicht selten kommt es zu konkurrierenden Ansprüchen bei Hallenzeiten, Personal und Finanzierungen.

\paragraph*{Der durch studentische Selbstverwaltung organisierten Hochschulsport}
Eine solche Form hat sich vor allem an kleineren Hochschulen bzw. Fachhochschulen etabliert, wo keine hauptamtlichen Sportlehrer/innen zur Verfügung stehen. Die Übernahme der Verantwortung durch Studierende begrenzt die Leistungsfähigkeit der Einrichtung, ermöglicht aber gleichzeitig den Erwerb von Kompetenzen für Studierende, die in anderen Organisationsformen nicht möglich sind. 

\section{Befragung ausgewählter Standorte}
Im Rahmen der Anforderungsanalyse an ein Hochschulsportverwaltungssystem wurde eine Befragung an acht Hochschulsporteinrichtungen durchgeführt zur Ermittlung existierender Systeme, wichtiger Kernfunktionen und Prioritäten (Tabelle~\ref{BefragungHochschulen}).

\begin{table}[h]
\centering
\resizebox{\textwidth}{!}{%
\begin{tabular}{|l|l|l|l|l|l|l|l|}
\hline
\textbf{Name}                     & \textbf{Bundesland} & \textbf{Typ} & \textbf{TN/Jahr} & \textbf{Mitarbeiter} & \textbf{Übungsleiter} & \textbf{Kurse/Jahr} \\ \hline
Hochschulsport Göttingen          & Niedersachsen       & übergreifend & 15.000           & 20                   & 150                   & 550                 \\ \hline
Hochschulsport Münster            & Nordrhein-Westfalen & übergreifend & 20.000           & 30                   & 500                   & n.a.                \\ \hline
Hochschulsport Osnabrück          & Niedersachsen        & übergreifend & 3.000            & 10                   & 100                   & 380                 \\ \hline
Hochschule Darmstadt              & Hessen              & exklusiv     & 1.000            & 2                    & 40                    & 120                 \\ \hline
Hochschulsport Uni des Saarlandes & Saarland            & übergreifend & 4.000/Woche      & 8                    & 150                   & 500                 \\ \hline
Hochschulsport Hamburg            & Hamburg             & übergreifend & 10.000/Pers.     & 15                   & 350                   & 1400                \\ \hline
Hochschulsport Berlin             & Berlin              & übergreifend & 18.000           & 20                   & 300                   & 1150                \\ \hline
Hochschulsport Kiel             & Kiel              & übergreifend & 3.000           & 20                   & 100                   & 300                \\ \hline
\end{tabular}}
\caption{Übersicht befragter Hochschulen}
\label{BefragungHochschulen}
\end{table}

Die Stichprobe stellt einen Ausschnitt unterschiedlicher Hochschulsporteinrichtungen bezüglich Teilnehmerzahl, Kursangebot, Bundesland und Mitarbeiter dar. Alle der befragten Einrichtungen gaben dabei an, als zentrale Einrichtung an der Universität bzw. Hochschule organisiert zu sein. Die Angaben zu den Teilnehmerzahlen im Jahr konnte nicht immer genau beantwortet werden, was auf folgende Gründe zurück zu führen ist:
\begin{itemize}
\item Die Teilnehmerzahlen können unterschieden werden nach Personenanzahl, Anzahl der Besucher je Kurs und Anzahl der Besucher je Kurstermin
\item Die verwendeten Systeme konnten keine genaue Aufschlüsselung darübermachen, wie viele Personen und Kursbesuche in einem Definierten Zeitraum stattfinden
\item Anmeldefreie Angebote werden in der Regel nicht mit Teilnehmerzahlen erfasst
\end{itemize}

Die daraufhin erfolgen Angaben sollen in diesem Fall als Schätz- und Näherungswerte dienen, um den Nutzerkreis etwa beurteilen zu können. 

Die Befragung gliederte sich in vier Teile:

\begin{enumerate}
\item Allgemeine Angaben zur Hochschulsporteinrichtung (siehe Tabelle ~\ref{BefragungHochschulen})
\item Was für eine System wird aktuelle verwendet und wie zufrieden ist der HSP mit der aktuellen Lösung
\item Bewertung der Wichtigkeit einzelner Funktionalitäten für den HSP
\item Zusätzliche Anmerkungen
\end{enumerate}

\subsection{Aktuelle Lösungen}
Um einschätzen zu können, mit Hilfe welcher Systeme die Organisation erfolgt, wurde ermittelt wie viele Einrichtungen eine spezielle Software für die Verwaltung des Hochschulsports nutzen. Der überwiegende Teil der Befragten mit 87.50 \% gab an eine spezielle Software zu verwenden und nur 12,50 \% nutzten eine andere Lösung in Form von Microsoft Excel und der Hochschulwebsite. Bei der Verwendung einer Hochschulsportverwaltungssoftware ergab sich ein Verhältnis von 71,43 \% zum System BuchSys und 28,57 \% mit der Software HSPinONE. Dies zeigt bereits die Verteilung der Systeme in der Hochschulsportlandschaft wobei das BuchSys-System noch weiter verbreitet erscheint, als schon angenommen. Die Verbreitung von HSPinONE ist dagegen als geringer einzuschätzen, da die Entwicklung abgebrochen wurde und diese Vertreter die einzigen Nutzer sind. Auf Nachfrage bei dem ehemaligen Entwickler, der Hochschulsport Marketing GmbH, wurde angegeben, dass der Bedarf an einem neuen System durchaus besteht, die Entwicklung aber nicht zur Ausrichtung des Geschäftsmodelles passt. Drei der sechs Befragten, die nicht HSPinONE nutzen gaben ebenfalls an, die Software als Alternative evaluiert und in für die Zukunft in Betracht gezogen hätten. Die befragte Hochschule ohne spezielles Verwaltungssystem befindet sich derzeit in der Evaluierungsphase, welche Software sich für eine Anschaffung eignet.
\\

Bei der Art des Betriebsmodelles ob On-Premise oder Software-as-a-Service (vgl. Abschnitt~\ref{SaaS}) gaben 75,00 \% der Befragten an das On-Premise Modell zu nutzen und 25,00 \% das SaaS Modell. Die Nutzer des On-Premise Modells entschieden sich in erster Linie aus Datenschutzrechtlichen Gründen für dieses Modell und gaben teilweise an, das keine Alternative zur Verfügung stand. Die Nutzung einer SaaS Lösung hat keine dieser Hochschuleinrichtungen bisher in Erwägung gezogen. Die Nutzer des SaaS Modelles entschieden sich aufgrund der einfache Wartung, unkomplizierte Umsetzung und die Anforderungen der Software für die Nutzung dieses Modelles.
\\
Die Nutzer der SaaS Lösung gaben an die Infrastruktur zu 100 \% durch den externen Dienstleister betreuen zu lassen und die Kosten entsprechend dafür zu tragen. Die Nutzer der On-Premis Lösung hingegen gaben alle an, das die Wartung und Instandhaltung für die Einrichtung ohne Kosten verbunden ist. Die Aufgabe der Betreuung wurde hingegen unterschiedlich verteilt. Hochschulsportmitarbeiter waren bei 83,3 \%, Hochschulrechenzentren bei 50 \% und externe Dienstleister bei immerhin 33,3 \% der Hochschulsporteinrichtungen an der Betreuung beteiligt. Die Aussagen zu den Betreuungskosten ist in diesem Zusammenhang zu hinterfragen, da sie sich in versteckten Kosten äußern kann. Sind Hochschulsportmitarbeiter in die Betreuung involviert, so fällt es in deren Arbeitszeit, die entsprechend vergütet wird und die Arbeitskraft in dieser Zeit nicht für andere Tätigkeiten zur Verfügung steht. Die Unterstützung des Hochschulrechenzentrums kann in diesem Fall als Unterstützung der Hochschule gesehen werden, so lange hier für den HSP keine Kosten anfallen.
\\

Wichtige Erkenntnisse sollte die Befragung im Fokus auf die Zufriedenheit liefern. Dazu musste auf einer sechsstufigen Skala eine Bewertung zwischen \textit{gar nicht zufrieden (1)} und \textit{sehr zufrieden (6)} gemacht werden. Die Hochschule ohne spezielles System soll hier nicht detailliert aufgeführt. Es zeigt sich jedoch dass nur in den Bereichen Kosten (zufrieden) und Ausfallsicherheit, Betreuung/Support (eher zufrieden) leicht positive Bewertungen gemacht werden konnten. 
\\
Untersucht wurde die Zufriedenheit im Hinblick auf die Aspekte Funktionsumfang, Anpassung der Darstellung, Anpassung der Logik, Dokumentation, Betreuung/Support, Weiterentwicklung, Kosten, Ausfallsicherheit, Performance und Bedienbarkeit (Abbildung~\ref{fig:Umfrage_zufriedenheit}).

	\begin{figure}[h]
		\centering
		\includegraphics[width=1\linewidth]{images/umfrage_zufriedenheit}
		\caption{Zufriedenheit mit der aktuellen Lösung}
		\label{fig:Umfrage_zufriedenheit}
	\end{figure}

Sehr zufrieden bis zufrieden wurden die Systeme in den Bereichen \textit{Kosten} ($\varnothing$ 5,86; \textpm 0,38), \textit{Ausfallsicherheit} ($\varnothing$ 5,57; \textpm 0,79), \textit{Performance} ($\varnothing$ 5,57; \textpm 0,79), \textit{Bedienbarkeit} ($\varnothing$ 5,43; \textpm 0,79) und \textit{Funktionsumfang} ($\varnothing$ 5,14; \textpm 1,07) bewertet. Die geringe Varianz bestätigt hier ein recht homogenes Bild in der Bewertung.
\\
Unterschiede in der Beurteilung werden bei den weiteren Aspekten deutlich. Die Anpassungsmöglichkeiten bezüglich \textit{Darstellung} ($\varnothing$ 4,14; \textpm 1,07) und \textit{Logik} ($\varnothing$ 3,83; \textpm 1,33) sowie die \textit{Betreuung/Support} ($\varnothing$ 4,43; \textpm 1,27) zeigen ein höhere Varianz auch wenn die durchschnittliche Wahrnehmung noch gut ist. Die größten Unterschiede sehen die Hochschulsporteinrichtungen jedoch bei den Themen \textit{Dokumentation} ($\varnothing$ 3,83; \textpm 1,94) und \textit{Weiterentwicklung} ($\varnothing$ 3,71; \textpm 1,25). Es ist an dieser Stelle zu bemerken, dass es sowohl zum System BuchSys als auch zum System HSPinONE keine offizielle Dokumentation gibt. Dies mag für HSPinONE im Hinblick auf die Beendigung des Projektes noch nachvollziehbar sein, für ein System wie BuchSys mit hoher Verbreitung ist es jedoch unverständlich. Das die Bewertung hier nicht negativer ausfällt erscheint überraschend, jedoch wurde von den Hochschulsporteinrichtungen angegeben, dass eine grundlegende Dokumentation für das BuchSys System besteht, die aber unter strengen Sicherheitskriterien nur für Kunden zugänglich gemacht wird.
\\
Als eine der kritischen Performance Stellen wurde das Verhalten des Systems zu Anmeldebeginn ausgemacht. Eine Anmeldung im Modus einer \textit{Race Condition} oder \textit{First come, first serve} unterliegt immer der Gefahr bei sehr nachgefragten Inhalten zur Startzeit sehr hohe Nutzerzahlen bedienen können zu müssen. Die Frage nach der Häufigkeit von Beeinträchtigungen zu Anmeldebeginn wurde von 42,86 \% mit gelegentlich, 28,57 \% mit selten und 28,57 \% mit nie beantwortet. Regelmäßige oder immer Beeinträchtigungen wurden nicht genannt. Eine Quantifizierung schwierig und der Grad der Beeinträchtigungsbeurteilung subjektiv geprägt. 71,4 \% der Befragten war jedoch schon von Problemen betroffen (vgl. Abbildung:~\ref{fig:Umfrage_beeinträchtigung}).
	\begin{figure}[h]
		\centering
		\includegraphics[width=0.7\linewidth]{images/umfrage_beeinträchtigung}
		\caption{Häufigkeit der Beeinträchtigung zu Anmeldebeginn}
		\label{fig:Umfrage_beeinträchtigung}
	\end{figure}

Die anschließend Frage ob dafür Anpassungen in der Anmeldung gemacht werden mussten, wurde von 71,43 \% der Einrichtungen mit nein und nur von 28,57 \% mit ja beantwortet. Aus methodischer Sicht ist die Frage in diesem Fall unglücklich gewählt, da in anschließenden Gesprächen eine sehr unterschiedliche Interpretation zur Beantwortung führte. So liegt es im Ermessensspielraum, ob die Auslagerung der Buchungen aus dem verwendeten Content Management System (CMS) zu Anmeldestart eine Anpassung am Anmeldeprozedere ist oder nicht. Ebenso verhält es sich mit der bewussten Staffelung von Anmeldungen über einen gewissen Zeitraum, um die Last zu entzerren.
\\
In der abschließenden Frage der allgemeinen Zufriedenheit antworteten sechs der acht Befragten mit ihrem System zufrieden zu sein.

\subsection{Bewertung der Wichtigkeit von Funktionalitäten}
Nachdem ein Überblick über die Bewertung der aktuell verwendeten Systeme gewonnen werden konnte, sollte im dritten Teil die Wichtigkeit einzelner Anforderungen und Funktionalitäten evaluiert werden. Dabei konnten wie Befragten wieder auf einer sechsstufigen Skala von $+++$ (1) bis $---$ (6) wählen. Zum Einstieg wurden erneut die nicht funktionalen Anforderungen abgefragt (Abbildung:~\ref{fig:Umfrage_anforderungen}).

	\begin{figure}[h]
		\centering
		\includegraphics[width=1\linewidth]{images/umfrage_anforderungen}
		\caption{Zufriedenheit mit der aktuellen Lösung}
		\label{fig:Umfrage_anforderungen}
	\end{figure}
	
Als wichtigste Faktoren wurden von allen Hochschulsporteinrichtungen die Aspekte \textit{Ausfallsicherheit} ($\varnothing$ 1,00; \textpm 0,00) und \textit{Sicherheit/Datenschutz} ($\varnothing$ 1,14 und \textpm 0,38) genannt. Dieses Resultat ist wenig überraschend, wurde doch der Datenschutz als entscheidendes Kriterium für die Auswahl eines Modelles genannt und durch die hohe Unterstützungsleistung im täglichen Geschäft ist eine höchst mögliche Ausfallsicherheit zwingend notwendig. Ebenfalls als sehr wichtig eingeschätzt wurden \textit{Performance} ($\varnothing$ 1,43; \textpm 0,79), \textit{Support/Betreuung} ($\varnothing$ 1,57; \textpm 0,53) und \textit{Weiterentwicklung} ($\varnothing$ 1,86; \textpm 0,38). Bemerkenswert ist, dass der \textit{Preis} mit $\varnothing$ 2,29 und \textpm 0,76 im Vergleich eine eher untergeordnete Rolle spielt. Deutlich unterschiedlich wird dagegen die Wichtigkeit der Bereiche \textit{Dokumentation} ($\varnothing$ 2,71; \textpm 1,25) und \textit{Individuelles Design} ($\varnothing$ 3,43; \textpm 0,98) wahrgenommen.
	
Eine anschließende Befragung sollte Aufschluss darüber geben, welche Endgeräte primär unterstützt werden sollten. Die Wichtigkeit von \textit{Web} $\varnothing$ 1,13; \textpm 0,35 und \textit{Mobil (Smartphone/Tablet)} $\varnothing$ 1,38; \textpm 0,74 wird als annähernd gleich wichtig gesehen während die \textit{Desktop Anwendung} mit $\varnothing$ 2,50; \textpm 2,00 in der Wichtigkeit sehr unterschiedlich beurteilt wurde. Dies lässt sich vermutlich darauf zurück führen, was als Desktop Anwendung gesehen wird. Ist der Webbrowser in dem die Internetseite dargestellt wird jetzt zum Web zuzuordnen oder zur Desktop Anwendung? Was ist wenn der Webbrowser auf einem mobilen Endgerät verwendet wird? Es lassen sich in diesem Fall Schwächen in der Fragestellung erkennen, jedoch wird ebenfalls deutlich, das eine Internet basierte Lösung von den meisten Befragten präferiert wird.


Die im Hinblick auf die Konzeption eines neuen Systemes wichtigste Frage beschäftigte sich mit der Einschätzung bestimmter Funktionalitäten. Die Befragten sollten 14 Funktionalitäten entsprechend ihrer Wichtigkeit beurteilen (siehe Abbildung \ref{fig:Umfrage_funktionen}).
	\begin{figure}[h]
		\centering
		\includegraphics[width=1\linewidth]{images/umfrage_funktionen}
		\caption{Wichtigkeit einzelner Funktionalitäten}
		\label{fig:Umfrage_funktionen}
	\end{figure}

Insgesamt zeigt sich bei vielen der abgefragten Funktionalitäten ein sehr heterogenes Bild und eine hohe Streuung. Dies ist wenig überraschend, so sind die Anforderungen und Bedürfnisse an jeder Hochschulsporteinrichtung doch sehr verschieden.
Insgesamt einig waren sich die Befragten bei der Wichtigkeit der Funktionen \textit{Kursverwaltung} und \textit{Teilnehmerverwaltung}. Sie wurden als elementar und mit der höchsten Wichtigkeit eingestuft ($\varnothing$ 1,00; \textpm 0,00). Diese Funktionen gehören zu den Eckpfeilern einer jeder Hochschulsporteinrichtung und ohne sie ist eine Verwaltung nicht möglich. Ähnlich verhält es sich mit den Übungsleitern, deren Verwaltung mit $\varnothing$ 1,50 und \textpm 0,76 ebenso hoch eingestuft wurden.
Alle anderen Funktionalitäten sind gekennzeichnet durch eine hohe Standardabweichung von \textpm 1,25 bis 2,00. Als wichtig eingestuft werden zusätzlich die \textit{Fitnessstudio} ($\varnothing$ 2,13; \textpm 1,25), \textit{Statistische Auswertung} ($\varnothing$ 2,50; \textpm 1,41), \textit{Verträge/Mitgliedschaften} ($\varnothing$ 2,50; \textpm 1,60), \textit{Automatisierte Zutrittskontrolle} ($\varnothing$ 2,50; \textpm 2,00) und \textit{granulares Rechtekonzept} ($\varnothing$ 2,63; \textpm 1,51). Die Varianz kann am Beispiel Zutrittskontrolle gut fest gemacht werden. Während für Einrichtungen mit entsprechenden Vorrichtungen (z.B. Drehkreuz) die Funktionalität sehr wichtig ist, so ist sie für diejenigen die nicht darüber verfügen eher zu Vernachlässigen.

Als weniger wichtig beurteilten die befragten Einrichtungen die Funktionalitäten wie \textit{Rechnungswesen} ($\varnothing$ 3,00; \textpm 1,93), \textit{Point of Sales} ($\varnothing$ 3,13; \textpm 1,46), \textit{Verschiedene Platzvergabe Modi} ($\varnothing$ 3,88; \textpm 1,96), \textit{Verleih} ($\varnothing$ 4,25; \textpm 1,49) und \textit{(Geschenk)Gutscheine} ($\varnothing$ 4,63; \textpm 1,30). Bei der Betrachtung gilt es jedoch zu beachten, das auch hier die Varianz meist sehr groß ist und diese Funktionalitäten für einige Einrichtungen doch eine deutlich höhere Wichtigkeit besitzen.
Einzig bei der Funktionalität \textit{Zeiterfassung für Mitarbeiter} ($\varnothing$ 5,50; \textpm 1,41) ist man sich in der Unwichtigkeit weitgehend einig.

Betrachtet man die Resultate dieser Frage, so bestätigt sich das bereits gewonnen Bild der unterschiedlichen Anforderungen und Herausforderungen der einzelnen Hochschulsporteinrichtungen. Funktionen die für einzelne Einrichtung sehr wichtig sind, können für Andere sehr unwichtig sein. Die sehr allgemeine Fragestellung lässt hier jedoch viel Interpretationsmöglichkeiten, wie die einzelnen Begrifflichkeiten belegt werden. So ist das Konzept der Gutscheine inzwischen weit verbreitet in Form von Geschenkkarten für zum Beispiel iTunes, Amazon, Google Play, Ikea, etc. Eine Etablierung im Hochschulsport Umfeld bedürfte jedoch einer zusätzlichen Anpassung im Geschäftsprozess, da die Rechnungsstellung darauf Rücksicht nehmen muss.

Betrachten wir die Ergebnisse unabhängig von der Bewertung der einzelnen Funktionen, so wird sehr deutlich, welche Flexibilität von einem Hochschulsportverwaltungssystem erwartet wird, wenn annähern alle Bedürfnisse berücksichtigt werden sollen. Um diese Anforderungen zu einigen Funktionalitäten noch weiter auszudifferenzieren wurden den Probanden drei weiter Fragen zu den Themenfeldern Point of Sales, Rechnungswesen (Bezahloptionen) und Fitnessstudio gestellt. Dabei ergaben sich folgende Ergebnisse:

\paragraph*{Fitnessstudio}
Die Bewertung der Anforderungen an spezielle Fitnessstudio Funktionen erwies sich als sehr stark abhängig von der Tatsache, ob in der Hochschulsporteinrichtung bereits ein Fitnessstudio betrieben wird bzw. eines in Planung ist. Ein separates Programm ($\varnothing$ 2,50; \textpm 1,97) war für die Probanden die wichtigste Funktionalität. 50 \% der Befragten sahen es als sehr wichtig an. Die Möglichkeiten einer Trainerbuchung bzw. Personal Trainings ($\varnothing$ 3,43; \textpm 1,62) und einer Warenwirtschaft ($\varnothing$ 4,00; \textpm 1,41) wurden deutlich weniger wichtig angesehen. 

\paragraph*{Rechnungswesen (Bezahloptionen)}
Im Rahmen einer Hochschulsporteinrichtung im Umfeld einer Hochschule haben sich viele Bezahloptionen etabliert. Die Veränderungen durch die Einführung von SEPA im Lastschrift Verfahren, hat diese Heterogenität weiter unterstützt und für Unsicherheit gesorgt. Bei der Frage wie wichtig derzeit die Unterstützung einzelner Optionen ist, lässt sich eine deutliche Abstufung erkennen. Das Lastschrift wird von allen Einrichtungen als zwingend notwendig erachtet ($\varnothing$ 1,00; \textpm 0,00). Dies liegt vor allem in der gelebten Praxis begründet. So ist diese Verfahren bekannt und lässt sich leicht an die Finanzabteilung der Hochschule auslagern. 
Weiterhin als wichtig wird zudem die Möglichkeit der Rechnungsstellung erachtet ($\varnothing$ 3,13; \textpm 2,23). Die hohe Standardabweichung zeigt jedoch schon hier, wie unterschiedlich die einzelnen Interessen gelagert sind. Dies setzt sich bei den Optionen Bargeld ($\varnothing$ 3,75; \textpm 1,98), Unikarte ($\varnothing$ 3,75; \textpm 2,12), EC-/Kreditkarte ($\varnothing$ 3,75; \textpm 1,75) und Online-Bezahlsysteme ($\varnothing$ 3,88; \textpm 1,96) weiter fort.

\paragraph*{Point of Sales}
Eine weitere Frage bezog sich auf die Unterstützung von Point of Sales (Verkaufs-)Plätzen. Ein PC basiertes Kassensystem ($\varnothing$ 2,88; \textpm 2,10) und die Unterstützung von Kartenlesegeräten ($\varnothing$ 2,88; \textpm 1,64) wurde im Mittel als wichtigste Unterstützungsleistung angesehen. Ein Barcode Scanner dagegen erscheint weniger wichtig zu sein ($\varnothing$ 3,63; \textpm 1,69). Die Unterstützung zusätzlicher Kassensysteme auf mobiler Basis ($\varnothing$ 4,00; \textpm 1,91) oder als Kassensystem von Drittanbietern ($\varnothing$ 4,86; \textpm 1,21) steht auf der Prioritätenliste der Befragten hinten an.

\subsection{Zusätzliche Anmerkungen}
Zum Abschluss des Befragung und des Interviews wurden die Teilnehmer nach sonstigen Anforderungen und Funktionalitäten gefragt. Folgende Funktionen wurden von mehreren Parteien als wichtig erachtet:
\begin{enumerate}
\item CMS Funktionalitäten oder Möglichkeit der Einbettung in ein CMS System
\item Sportstättenverwaltung
\item Umfangreiche Mailing Funktionalitäten
\item 
\end{enumerate}


%\section{Wichtige Geschäftsprozesse}
%\begin{enumerate}
%\item Kunden anlegen
%\item Kunden verifizieren
%\item Kurse anlegen
%\item Kurse finden
%\item Kurse darstellen
%\item Kurse buchen
%\item Geld einziehen
%\item Rechnungsstellung
%\item Teilnehmer informieren
%\item Übungsleiter abrechnen
%\item Sportstätten vermieten
%\item Verträge verwalten
%\item Zutritt überprüfen
%\item 
%\item 
%\end{enumerate}
\section{Probleme und Herausforderungen}