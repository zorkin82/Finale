\chapter{Analyse der Fachdomäne Hochschulsport} 
\label{ch:hochschulsport} 
Die Aufgaben und Anforderungen, die an deutsche Hochschulen gestellt werden sind vielfältig und erstrecken sich in viele verschiedene Bereiche. Neben den Kernaufgaben universitäre Lehre und wissenschaftliche Forschung gehört es auch zu den Aufgaben das studentische Leben in sozialen und gesellschaftlichen Belangen zu unterstützen und fördern. Neben Einrichtungen wie dem Studierendenwerk oder Allgemeinen Studierendenausschuss (AStA) ist auch der Hochschulsport eine Einrichtung die unterschiedlichste Aufgaben im Umfeld der Fachhochschulen und Universitäten übernimmt. Dieser Teil soll ein Verständnis für die Domäne Hochschulsport und deren Entwicklung, Organisationsformen, Aufgaben sowie Probleme und Herausforderungen vermitteln.

\section{Historische Entwicklung des Hochschulsports}
Das Sporttreiben an deutschen Hochschulen begann zwar nicht erst nach dem zweiten Weltkrieg, die historische Betrachtung soll sich hier jedoch auf die Zeit zwischen 1948 und 2009 beschränken, um den Rahmen der Arbeit nicht zu sprengen. Da die Organisation von Bildung in die Zuständigkeit der Bundesländer fällt, hat die Entwicklung durchaus unterschiedliche Wege genommen. Ich konzentriere mich in diesem Fall auf den geschichtlichen Rückblick mit dem Schwerpunkt NRW, da hier die umfangreichsten Dokumentationen verfügbar waren. 
Der 1999 von von 29 Bildungsministern initiierte Bologna-Prozess zur Schaffung eines einheitlichen Hochschulraums (vgl. QUELLE)
\\ \\

Hochschulsport heute\\
Durch die gesetzliche Verankerung im Hochschlrahmengesetz und die Festschreibung in den einzelnen Landeshochschulgesetzen hat sich eine vorerst gesicherte Grundlage zur Förderung des Sports an Hochschulen manifestiert. Die einzelnen Hochschulsporteinrichtungen besitzen eine gefestigte Position im Hochschulkontext und müssen nicht mehr um ihre grundlegende Daseinsberechtigung kämpfen. 
Das Grundanliegen des HSP ist und bleibt jedoch die Förderung des Sporttreibens der Studierenden, die sich in speziellen Lebensphasen befinden und durch besondere Umstände gekennzeichnet sind. Das diese nicht generalisiert werden können, lässt sich leicht nachvollziehen, da unterschiedliche Studiengänge, Wohnumstände und Studiumsformen die Lebensphase in sehr heterogener Weise beeinflussen können. Der Einfluss auf den Hochschulsport und im speziellen auf die Akzeptanz und Teilnahme lässt sich an folgenden Kriterien festmachen, die speziell bei der Hochschulsportorganisation explizit berücksichtigt werden müssen:

\begin{enumerate}
	\item Sportinteresse des Studierenden
	\item Ort der Sporteinrichtungen
	\item Soziale Kontakte
	\item Belastungen und Entbehrungen
\end{enumerate}

Somit ergibt sich für den Hochschulsport heute die besondere Herausforderung den Bedürfnissen der Studierenden gerecht zu werden und ihnen die Gelegenheit zu geben, gemäß ihrer Interessen und Vorstellungen Sport treiben zu können. \\
Der bereits beschriebene Bologna-Prozess führte allerdings zu zusätzlichen Herausforderungen im Verhältnis Hochschulsport-Studierende. Die Verkürzung der Studienzeit führte zu einem strenger vorgegebenen Studienplan und einer deutlichen Verknappung der eigenen Gestaltungs- und Freizeitmöglichkeiten. Diese Faktoren beeinflussen die Möglichkeiten der Studierenden an der Teilnahme im Hochschulsport erheblich, sodass sich des Hochschulsportprogramm diesen speziellen Anforderungen noch stärker durch mehr Flexibilität und neue Angebotsformen stellen muss.
\\

Neben den Studierenden gehören aber zunehmend auch andere Personenkreise zur Zielgruppe des Hochschulsports. Besonders die Bediensteten der Hochschule werden zunehmend in den Programmen mit speziellen Programmen angesprochen. Speziell im Bereich Gesundheitssport haben sich hier eigene Programme an den Hochschulen oder auch durch übergeordnete Gremien, wie den allgemeinen deutschen Hochschulsportverband, etabliert. Das Angebot ist dabei sehr unterschiedliche und stark abhängig von den verfügbaren Ressourcen, erstreckt sich aber von der Teilnahme am allgemeinen Programm über spezielle Bediensteten Kurse bis hin zur individuellen Betreuung am Arbeitsplatz im Büro. Die Nachfrage nach Gesundheitsfördernden Maßnahmen dieser Art erfreut sich derzeit einer hohen Nachfrage. \\
Neben den Bediensteten lassen sich aber auch Externe Teilnehmer als weitere Zielgruppe bestimmen. Die Einbindung dieser Personenkreise unterscheidet sich am stärksten in den Unterschiedlichen Einrichtungen und birgt großes Potenzial und Gefahr gleichzeitig. Je nach Organisationsform stehen dem Hochschulsport ganz unterschiedliche Möglichkeiten offen aber auch das regionale Umfeld muss hier genau in Betracht gezogen werden. Mit der Öffnung des Hochschulsports für externe Teilnehmer begibt man sich in direkte Konkurrenz mit kommerziellen Anbietern. Die Integrationsformen variieren von Kooperationen mit Vereinen, Firmen im Rahmen von Betriebssportangeboten, Rehakliniken und Gesundheitszentren bis hin zur freien Öffnung für Jedermann. Allgemein lässt sich jedoch beobachten, dass diese Gruppe vor allem dann angesprochen wird, wenn eine Auslastung durch Bedienstete und Studierende nicht möglich ist. Im Bereich "Organisationsformen" werde ich auf die Unterschiede noch detaillierter eingehen.  


\subsection{Typische Aufgaben}
\subsection{Organisationsformen}
Die zentralen Einrichtungen\\
die – unabhängig von einem Fachbereich und direkt dem Senat unterstellt – den Hochschulsport eigenständig organisieren und verwalten
\\ \\
Die in das Aufgabenfeld der Institute für Sportwissenschaft integrierte Organisation des Hochschulsports \\
(diese enge Verbindung mit der Sportlehrer/innenausbildung kann sich auf das Hochschulsportangebot äußerst positiv auswirken, bringt aber auch durch notwendige „Prioritätensetzungen“ die Gefahr der Benachteiligung mit sich);
\\ \\
Der durch studentische Selbstverwaltung organisierten Hochschulsport \\
(diese Form hat sich vor allem an kleineren Hochschulen bzw. Fachhochschulen etabliert, wo keine hauptamtlichen Sportlehrer/innen zur Verfügung stehen)
\cite[vgl.][]{Radde.1996}


\subsection{Befragung ausgewählter Standorte}
\subsection{Wichtige Geschäftsprozesse}
\begin{enumerate}
\item Kunden anlegen
\item Kunden verifizieren
\item Kurse anlegen
\item Kurse finden
\item Kurse darstellen
\item Kurse buchen
\item Geld einziehen
\item Rechnungsstellung
\item Teilnehmer informieren
\item Übungsleiter abrechnen
\item Sportstätten vermieten
\item Verträge verwalten
\item Zutritt überprüfen
\item 
\item 
\end{enumerate}
\subsection{Probleme und Herausforderungen}